\chapter{Conclusion}
\label{ch:conclu}

\newthought{Il n'est pas} évident de conclure sur un tel sujet, ne serait-ce que
parce des changements peuvent se produire à une vitesse insoupçonnée (comme ce
fut le cas avec le Freedom Act), et que des conclusions vraies aujourd'hui
s'avéreront sans doute fausses demain.

\newthought{Quoi qu'il en soit,} certains choses sont aujourd'hui claire
concernant ces programmes de surveillance : quand ils ne sont pas purement
illégaux, ces derniers bafouent allégrement les libertés individuelles de tout
un chacun, au nom d'une sécurité somme toute relative, puisque l'efficacité de
ces programmes est remise en cause par tous les rapports (cela n'empêche pas la
France de voter une Loi sur le Renseignement qui sera inapplicable
techniquement\ldots)

\newthought{Sans le savoir,} et sans le mériter, des millions de personnes ont
perdu ces dernières années une partie de leur liberté, afin de satisfaire aux
délires des sécuritaires de nos élites, qui ne pouvaient pas ne pas être au
courant des agissements des services (quand ils ne s'occupent pas directement de
la chose\ldots En effet, Ziad Takieddine, proche de Nicolas Sarkozy, à négocié
directement la vente de l'Eagle d'Amesys avec Kadhafi\ldots)

\newthought{Ces programmes,} aux dérives caractéristiques des régimes
totalitaires, ne peuvent en aucun cas être acceptés sous couvert de la sécurité, ou de tout autre
prétexte, ne serait-ce que parce qu'elles sont contraires aux valeurs des
Nations qu'ils prétendent protéger.  Dans leur forme actuelle, ils sont
dangereux pour les libertés individuelles et pour la survie de nos régimes
démocratiques.
