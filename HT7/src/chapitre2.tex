%----------------------------------------------------------------------------------------
%	CHAPTER 2
%----------------------------------------------------------------------------------------

\chapter{Quis custodiet ipsos custodes ? -- Qui gardera les gardes ?}
\label{impact}

~\vfill
\begin{doublespace}
\noindent\fontsize{18}{22}\selectfont\itshape
\nohyphenation
\openepigraph{<< La seule chose qui permet au mal de triompher est l'inaction
des hommes de bien. >>}{-- Edmund Burke}
\end{doublespace}
\vfill
\vfill

\newthought{Les services de renseignement} sont une émanation du pouvoir
politique, et sont sensés être sous son contrôle, et donc sous le contrôle
indirect du peuple. Cet état de fait rend donc assez difficilement
compréhensible la boulimie d'informations dont font preuve les services à
l'encontre même de leurs  concitoyens.

\newthought{Outre l'inefficacité} flagrante de ces programmes\autocite{surve}
(la débauche de programmes différents aux \EUA~n'a par exemple pas permises de
déjouer les attentats de Boston, pas plus que les écoutes en France n'ont permis de déjouer
les attentats de janvier contre Charlie Hebdo), c'est leur aspect légal qui est
aujourd'hui remis en question.

\newthought{En effet,} bien qu'adossés au Patriot Act et FISA, les programmes
de surveillance américains sont contestés légalement, d'une part par les pays qui
en sont <<~victimes~>> ou qui voient leurs communications écoutées, d'autre part
par certains élus locaux, qui estiment que l'automatisme caractérisé desdits
programmes empêche de facto tout contrôle humain, et donc rend la surveillance
incontrôlable.

\newthought{Il faut bien} comprendre que l'aspect massif de ces programmes
nécessite forcément de faire appel à d'autres algorithmes de surveillance, qui
n'ont pas une efficacité de 100\% (et ne s'en rapprochent même pas d'ailleurs).
Le contrôle humain est ainsi très limité, et ne peut de toutes façons que
concerner une infime portion de ce qui est réellement récolté. Quelle instance
de contrôle pourrait vérifier des dizaines de milliards de données par mois ?
Qui peut certifier que dans ce tas de données ne figurent pas des choses qui ne
sont pas légalement sensées être là ? A tout hasard, des communications de nos propres
citoyens\ldots 

\newthought{La France a} même réussi à faire encore plus fort que ça, en
légalisant \emph{a posteriori} des dispositifs mis en place depuis des années,
présents sur le territoire français, alors que ces derniers n'étaient encadrés
par aucun texte jusqu'au récent vote de la Loi sur le
Renseignement.\autocite{mediapart} En effet, malgré l'existence d'une
commission chargé du contrôle des interceptions de sécurité (la CNCIS), une
grande partie des écoutes opérées sur le territoire nationale se retrouvent
hors de sa portée, et donc hors de tout contrôle externe à la hiérarchie des
services.

\subsection{L'avis du Conseil de l'Europe}

\newthought{Après les premières} révélations d'Edward Snowden et les légitimes
questions qu'elles soulèvent, le Conseil de l'Europe a commandé un rapport
étudiant la légalité de ces dispositifs au regard du droit européen.

\newthought{Un an plus tard,} soit début janvier 2015, ce rapport est
publié\autocite{rapport} et dénonce de façon très vive et très critique les
pratiques de la NSA, et donc de facto celles des autres services, y compris européen. Même les plus
conservateurs des membres de ce Conseil ont ratifié le texte en l'état, sans
amendement. 

\newthought{Cet article} est très clair quant à la nature complètement illégale
de ces programmes : <<~Ces opérations mettent en danger les droits de l'homme
fondamentaux, notamment le droit au respect de la vie privée […], le droit à la
liberté d'information et d'expression […], le droit à un procès équitable et à
la liberté de religion.~>> Ce type d'espionnage serait, toujours selon ce
rapport, une violation de la Convention européenne des droits de l'homme et
de la convention du Conseil de l'Europe sur la protection des données
personnelles\ldots 

\newpage
\newthought{A noter que} pour la rédaction de ce rapport, ni les services
américains, ni, et c'est plus surprenant, les services européens ont accepté de
collaborer et/ou d'être entendus par les rapporteurs. Si la chose est
parfaitement compréhensible pour les services américains, qui ne sont pas tenus
de rendre des comptes à une puissance étrangère, l'attitude des services
européens est quant à elle bien plus étrange, et prouve qu'un véritable malaise
juridique règne quant à la légalité de ces programmes.

\newthought{Le rapport s'en} prends même indirectement à la classe politique :
<<~L'emballement de cette machine de surveillance est dû au fait que les
dirigeants politiques ont perdu le contrôle des activités des services de
renseignement.~>>, ce qui est parfaitement en phase avec les révélation faites
par Mediapart, décrites plus haut, sur la perte de contrôle de la CNCIS en
France.

\newthought{Il faut bien} comprendre qu'un citoyen peut se retrouver dans
plusieurs cas de figure, vis à vis de ces programmes :

\begin{itemize}
  \item Un citoyen normal ne saura jamais si ses données ont pu être
  interceptées et exploitées, ni sur quelle base cela fut demandé, alors que
  c'est pourtant son droit le plus strict. Après tout, <<~Notre liberté repose
  sur ce que les autres ignorent de notre existence.~>>\footnote{Citation tirée
  du rapport.}
  \item Un citoyen curieux n'a actuellement aucun recours, aucune instance
  juridique vers laquelle se tourner si il soupçonne que ses données aient pu
  être interceptée sans raison légale.
  \item Un citoyen soucieux de son droit à la vie privée n'a aujourd'hui que
  très peu d'options techniques pour échapper à cette surveillance aveugle, la
  dissymétrie de moyens mis en place par les services (financiers et techniques)
  étant trop importante.
\end{itemize}

\subsection{Freedom Act}

\newthought{Aux \EUA}, suite à de nombreuses pressions juridiques (à travers des
organismes comme l'ACLU\footnote{American Civil Liberties Union} et
l'EFF\footnote{Electronic Frontier Foundation}) et populaires, les lignes
commencent à bouger. Le récent vote du Freedom Act, sensé rétablir des contrôles
et empêcher toute collecte aveugle, a été obtenu de haute
lutte\autocite{freedom}.

\newthought{Même si certains} pensent que ce texte ne va pas assez loin et
aurait du exiger une abrogation complète de ces programmes et une réforme des services de
renseignement, ce texte fut accueilli plutôt positivement par les opposants à la
surveillance de masse.


