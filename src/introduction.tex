\chapter{Introduction}
\label{ch:intro}

Quiconque n'a pas vécu en autarcie complète ces derniers
mois a forcément entendu parler des révélations faites par Edward Snowden sur les
différents programmes mis en place par les agences de renseignement américaines
pour surveiller une grande partie de la population mondiale, y compris leur
propres concitoyens.\newline

Ces révélations n'ont pas surpris beaucoup de monde dans le milieu
de la sécurité informatique, de l'hacktivisme, et de la défense des droits des
citoyens en général. En effet, depuis les premières révélations sur le
programme ECHELON en 1988 par Duncan Campbell\cite{Campbell1988}, il est de
notoriété publique que les \EUA~et leurs alliés des Five Eyes\footnote{Le
Royaume-Uni, le Canada, l’Australie et la Nouvelle-Zélande}, dans le cadre du
traité UKUSA, ont mis en place une infrastructure à grande échelle d'écoute du
spectre électromagnétique (signaux satellites, GSM, fibre optique, micro-ondes,
etc).\newline

Ce qui n'était pas connu jusqu'alors, c'était l'ampleur de la
collecte de renseignements, et surtout qui était concerné par ladite collecte.
La réponse courte qu'on peut aujourd'hui donner est : \emph{tout le monde} ! Via
des accords bilatéraux, les agences de renseignement occidentales ont accès aux
méta-données\footnote{A la différence des données, les méta-données ne
sont que les <<~enveloppes~>> des communications : qui parle avec qui,
quand, où, etc.} de toute personne connectée, même leurs propres citoyens
(alors qu'un service d'espionnage extérieur ne peut normalement pas collecter
des données sur ses propres citoyens, mais nous y reviendrons à la
page~\pageref{impact})\newline

Cette collecte aveugle n'est pas sans poser quelques soucis à nos
principes démocratiques fondamentaux, comme par exemple le droit à la correspondance
privée\citep{CEDH}, et c'est ce que nous allons voir dans ce rapport.
